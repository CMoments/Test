\documentclass[10pt,hyperref={CJKbookmarks=true}]{beamer}
\usepackage[utf8]{inputenc}

\usetheme{Berlin}
\usepackage{xeCJK}
\usepackage{ctex}
\usecolortheme{default}

%------------------------------------------------------------
%This block of code defines the information to appear in the
%Title page
\title[第十六次直播课] %optional
{第十六次直播课}

\subtitle{习题讲解}

\author[李嘉政] % (optional)
{李嘉政}


\date[VLC 2021] % (optional)
{Dec 2023}

%End of title page configuration block
%------------------------------------------------------------



%------------------------------------------------------------
%The next block of commands puts the table of contents at the 
%beginning of each section and highlights the current section:

\AtBeginSection[]
{
  \begin{frame}
    \frametitle{Table of Contents}
    \tableofcontents[currentsection]
  \end{frame}
}
%------------------------------------------------------------


\begin{document}

%The next statement creates the title page.
\frame{\titlepage}


%---------------------------------------------------------
%This block of code is for the table of contents after
%the title page
\begin{frame}
\frametitle{Table of Contents}
\tableofcontents
\end{frame}
%---------------------------------------------------------

\section{获胜的概率2}

\begin{frame}{Solution}

设 $f_{i,j}$ 表示第 $i$ 轮在 $j$ 手上的概率,注意到转移可被矩阵表达,矩阵快速幂即可。时间复杂度 $\mathcal O(n^3\log k)$。

\end{frame}
\section{加kIII}

\begin{frame}{Solution}

设 $cnt_{i,j}$ 表示第 $i$ 轮,数字 $j$ 的出现次数,注意到转移是矩阵形式,矩阵快速幂即可。时间复杂度 $\mathcal O(Td^3\log k)$,其中 $d=10$。

\end{frame}
\section{球形空间产生器}

\begin{frame}{Solution}

对球体方程差分后,就是简单的线性方程组,高斯消元即可。时间复杂度 $\mathcal O(n^3)$。

\end{frame}
\section{行列式求值}

\begin{frame}{Solution}

消成上三角矩阵即可,时间复杂度 $\mathcal O(n^3)$。

\end{frame}
\section{阶乘的约数和}

\begin{frame}{Solution}

约数的和考虑每个质因子贡献即可,即若 $n=\prod p_i^{e_i}$,则 $n$ 的因数和为 $\prod \frac{p_i^{e_i+1}-1}{p_i-1}$。每个质因子在 $n!$ 中的出现次数是容易 $\mathcal O(\log n)$ 之内求出的,随便一种筛法都可以求出所有质因子,所以总时间复杂度为 $\mathcal O(n+C(n))$,$C(n)$ 为选择的筛法时间复杂度。

\end{frame}
\section{女巫的数字法阵}

\begin{frame}{Solution}

上课现场做。

\end{frame}
%\section{字符串的变换计数}

\begin{frame}{Solution}

等价于本质不同循环串个数。简单做法就是倍长后,Hash 判断即可。时间复杂度 $\mathcal O(n\log n)$。

\end{frame}
%\section{最优操作}

\begin{frame}{Solution}

一个显然的贪心策略是,每次选择最大的操作数,把最小的数替换掉。堆即可。时间复杂度 $\mathcal O(n\log n)$.

\end{frame}

%\section{建造国家}

\begin{frame}{Solution}

容易证明答案一定是 $k-1$.

\end{frame}
%\section{依依的瓶中信}

\begin{frame}{Solution}

插入 Trie 的时候同时判断匹配长度,正反各走一遍即可知道所有匹配情况。时间复杂度 $\mathcal O(\sum |S_i|)$。

\end{frame}
%\section{小蓝的神秘图书馆}

\begin{frame}{Solution}

上课现场做。

\end{frame}
% \section{糖果}

\begin{frame}{Solution}

$f_{i,S}$ 表示处理了前 $i$ 件物品,组成的糖果集合为 $S$ 时的最小物品数。转移只需枚举当前物品选不选即可。时间复杂度 $\mathcal O(n2^m)$。
 
\end{frame}
% \section{愤怒的小鸟}

\begin{frame}{Solution}

注意到杀死猪的顺序不影响结果。考虑 $f_{S}$ 表示杀死猪的集合为 $S$ 的最少鸟数,显然的转移即是要么只为杀一头猪,要么存在某条抛物线杀了多头猪。注意到该题中抛物线经过原点,所以只需要两点即可确定一条抛物线。而杀猪顺序无关,所以不妨每次枚举通过不在 $S$ 中的某个点的抛物线,代码取了最小的零。解抛物线就是简单的解线性方程。时间复杂度 $\mathcal O(n2^n)$。 

\end{frame}
% \section{小蓝的选边}

\begin{frame}{Solution}

$f_S$ 表示选择的点集为 $S$ 的最大价值和,转移就是枚举边即可。时间复杂度 $\mathcal O(n^22^n)$。

\end{frame}


% \section{First section}

% %---------------------------------------------------------
% %Changing visivility of the text
% \begin{frame}
% \frametitle{Sample frame title}
% This is a text in second frame. For the sake of showing an example.

% \begin{itemize}
%     \item<1-> Text visible on slide 1
%     \item<2-> Text visible on slide 2
%     \item<3> Text visible on slides 3
%     \item<4-> Text visible on slide 4
% \end{itemize}
% \end{frame}

% %---------------------------------------------------------


% %---------------------------------------------------------
% %Example of the \pause command
% \begin{frame}
% In this slide \pause

% the text will be partially visible \pause

% And finally everything will be there
% \end{frame}
% %---------------------------------------------------------

% \section{Second section}

% %---------------------------------------------------------
% %Highlighting text
% \begin{frame}
% \frametitle{Sample frame title}

% In this slide, some important text will be
% \alert{highlighted} because it's important.
% Please, don't abuse it.

% \begin{block}{Remark}
% Sample text
% \end{block}

% \begin{alertblock}{Important theorem}
% Sample text in red box
% \end{alertblock}

% \begin{examples}
% Sample text in green box. The title of the block is ``Examples".
% \end{examples}
% \end{frame}
% %---------------------------------------------------------


% %---------------------------------------------------------
% %Two columns
% \begin{frame}
% \frametitle{Two-column slide}

% \begin{columns}

% \column{0.5\textwidth}
% This is a text in first column.
% $$E=mc^2$$
% \begin{itemize}
% \item First item
% \item Second item
% \end{itemize}

% \column{0.5\textwidth}
% This text will be in the second column
% and on a second tought this is a nice looking
% layout in some cases.
% \end{columns}
% \end{frame}
%---------------------------------------------------------


\end{document}